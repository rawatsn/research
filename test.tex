\documentclass{article}
\begin{document}
INTRODUCTION - Why did we use provenance ?

Why provenance, where provenance gives us the information about the where the data is coming from and what is happening to the data. In this paper we attempt to use provenance to test our thoery that using provenance the process of calculating any metric over data set can be done in a shorter period of time there by increasing the profermance. This method not only increases the security of the model but also we do not have to normalize the dataset anymore.Provenance helps us effeciantly calculate the information gain over the shared dataset and adds accuracy as all the data is hashed and we calculate calculate information gain over those hash functions.
Here is an example :

T0 = owned data 

T1 = Shared Data 

T-int = Integrated data

T0 owned

\begin{tabular}{ |c|c|c| } 
 \hline
 Name & Age & Salary \\ 
\hline
 xyz & 21 & 40k \\ 
 \hline
\end{tabular}

T1 shared 

\begin{tabular}{ |c|c|c| } 
 \hline
 Name & Age & Salary \\ 
\hline
 xyz1 & 23 & 45k \\ 
 \hline
\end{tabular}

T-int

\begin{tabular}{ |c|c|c| } 
 \hline
 Name & Age & Salary \\ 
\hline
 xyz1 & 21-23 & 40-45k \\ 
 \hline
\end{tabular}

As for the provenance annotations they will be strings of 1s and 0s so we will have 2 power n possibilities and the distances will be  calculated over those annotations. The reason we chose this method was because every different string needs different annotation. The mertic of the distance we chose was hamming distance because it tells us the number of bits different from the other string and that is considered as a distance.
As for the hash function it self we are looking for something that converts strings to a string of 1s and 0s.
The other things we are looking at currently is to give all the similar values same hash values as it would increase the performance immensely so this we will have to use another metric called jaccard coefficient which gives us the similarity between two strings. If the J-metric is greater than a certain threshhold then the values are considered similar and we can give them same hash function. We need to test this theory for hash value collisions.


------------------------------------------------------------------------

Method 1 we looked at was to Ignore the T1 (Shared Dataset) as we cannot look at the Shared Dataset and Trust the integration process. So, our assumption is the T-integrated dataset is true or valid. 
So now we have Gain/Loss metric. This can be a ratio. All the same values / old values. For example, in the above tables we have new values like Name (xyz1 + 1), age (23 +1), salary (45k +1)

A = xyz
B = xyz1
We can calculate similarity measure (Jaccard Index) between names. Jaccard Index =
(A integration B) / (A union B). 
A integration B = xyz = 3
A union B = xzy1 = 4
JI = ¾ = 0.75
We can take a critical value if JI > 0.70 so if this is true both names are same. 
Case when there is only 1 extra character.
A int B = xyz = 3
A U B = xyz1 = 4
JI = ¾  = 0.75	Case when A = xyz  and  B = xy1
A int B = 2
A U B = 4
JI = 2/4 = 0.5
Case of xyz and xyz12
A = xyz and B = xyz12
JI = 3/5 = 0.6 	Case of A = Shemon and B = Simon
A int B = 4
A U B = 7
JI = 4/7 = 0.571
SRawat and Shemon Rawat
	4/6 = 2/3 = 0.67
Looking at the table above we can see when we have 1 extra character the similarity is 0.75
When we have 1 different character then JI = 0.5
If you have 2 extra characters, then JI = 0.6 
If bigger name and 1 different character, we get JI = 0.7143
So we can set critical value to 0.70
If JI is > 0.70 then it’s a similar value and if its less than that we can use clipping and calculate the SED between two names.

Going back to method 1 we can say both names xyz and xyz1 are same. So now we have 3 same values (xyz, 21, 40k) and we have 8 total values. And age (23) and salary(45k) 2 different values.
Same/different ration = 3/2 = 1.5
Different/same ratio = 2/3 = 0.67
Same/total = 3/8 = 0.375
Different/total = 2/6 = 1/3 = 0.333

What do they mean?

Method 2: ED can be calculated for all the integer values and SED can be calculated for String values. But SED and ED are not on the same metric. How to do that? Later we do it with log methods and use clipping on it. 

What if we apply ED for int values and SED for int values and then apply the log scaling on both of them so if SED values = 2,4,8,3,5,… after taking log values it scales down. We get = 1,2,3,1.58, 2.32, (some small values.)

If ED values = 200k,400k,800k they would look like log2 200 = 7.64, log2 400 = 8.64, log2 800 = 9.67. The values scale down from 800 to 9.67.

These metrics are much more comparable than before. 

Normalizing Method:
1.	Min Max Normalizing:
X’ = (x – x(min))/(x(max) – x(min))
In 1. We need the UB and LB. Its good for comparing variables like age and salary.
2.	Feature clipping: This method helps with outliers a lot. We can fix a max value and if a value is greater than the critical value then we can just set it to something. 
3.	Log scaling => x’ = log(x)
This is good when a dataset has very large and very small values. So to make the graph more linear we use log scaling.
4.	Z-score values: means the number of SDs, a value is away from mean. Z-score = 
(x-mean) /SD
5.	Data standardizations


Revising Entropy.
xxxxx xxxxx  if all the values are same i.e 10xs then entropy = 0.
If we get some o in that for example xxxxx oooxx then we increase entropy in that case. 
Entropy = -sum(p * log(p))
P = probability of that event happening.
Example : xxx ooo ---- 
N = 10
X = 3
O = 3
“-” = 4
Entropy = -3/10 log(3/10) – 3/10log(3/10) – 4/10log(4/10)
If we calculate entropy for all same (xxxx) then entropy = -1*log(1) = 0
For information gain we split the dataset. We canulate entropy for both the splits we find the entropy for the split data. We find the weighted entropy. Then IG = 1-wt entropy.

Weighted entropy always lower than the entropy of the entire dataset.

Cell level IG? (how? Using Provenance ?)
This means if I look at one cell or the entire data point and use a formulae on both we will should get same IG.
Taking T1, T0 and T-int from above we have 6 different values, 2 same values, 8 total values.
(xyz,21.40k),(xyz1,21-23,40-45k)
1.	 2.   3.     4.      5.   6.   7.   8.    

We can come up with a formula : sum (log (x-y)) = log 1 + log 2 + log (5000) = 0+1+12.287
Instead of using 5000 we can use 5 so log 5 = 2.32

Provenance: 
R(owned)
A	B	Prov symbol
1	a	Pa Pb 
2	b	Qa Qb
1	c	Ra Rb

S(Shared)
B	C	Prov symbol
a	10k	Xb Xc
c`	10k	Yb Yc
b`	20k	Zb Zc

R join S
A	B 	C	Symbol
1	A	10k	Pa(Pb+Xb)Xc
2	b+c`	10k	Qa(Qb+Yb)Yz
1	c+b`	20k	Ra(Rb+Zb)Zc

PX -> Pa.Pb.Xc OR Pa.Xb.Xc

Provenance increases Security and speed
(Pb + Xb) are same in this case => here + means =
(Qb + Yb) are different => so we store the difference so b-c`
(Rb + Zb) are different => so we store the difference so c-b`

If these values like these (Qb + Yb) contain range values and we can directly store the differences in them and display only the symbols for  | Qa | Qb + Yb |(b -c`) | Yz | then this will increase the security and the performance and we can use provenance using IG.

*log of 1 is 0. So if the difference between two strings is only 1 character then they are basically same values as so we get log1 = 0. As log scales the graph down and tries to make it linear this might help us to compare SEDs and EDs. 
We might need do a min max normalization together with log scaling in some order.
*Other thing is what if it’s a completely new value so for SED it will be length of the string OR just 1. For EDs we will calculate the distance from 0.

*we can use clipping for outliers , log scaling and min max together,
If we use log out IG will range from 0 – 100
As 2 power 0 = 1
And 2 power 10 = a very large number

Bloom Filter

2 methods :
1.	Hash values as provenance annotation. Eg 1 -> 0001 -> (R,A,1)
2.	Apply SED to all and the add ED

GOAL : to scale down bigger values to make graph linear.

https://www.usenix.org/conferences/author-resources/paper-templates
D(a,1)
R(1,b)
(Alb)
(a,b)
D(a,1) -> p
R(1,b) -> q
res(a,b) -> pq
a -> p
1 -> q
1 -> r
b -> s
(a,b) -> p(q=r)s
D(a,1) -> pq
IG -> s
a -> p
1 -> q
b -> s
D(a,1)
R(1,b), R(1,c)
res(alb)
res(a,b), res(a,c)
a -> p
1 -> 1
1 -> q
b -> r
c -> s
res(a,b) -> p(q=q)r
res(a,b) -> p(q=q)r -> pqr
res(a,c) -> p(q=q)s -> pqs
D(a,1) -> pq
R(1,b,c), R(1,c,b)
res(a,b,c), res(a,c,b)
res(a,b,c) -> pqrs
res(a,b,c) -> pqrs -> p(q=q)rs
(d.B = r.B);
D(A,B), R(B,C,D)
JOIN R r ON (d.B = r.B);
res: result table
D(a,1), res(a,b,c)
D(a,1) -> pq
res(a,b,c) -> pqrs
res(a,c,b) -> pqsr
{rs, sr}
{}
A = {pq}
B = {pqrs, pqsr}
A = {}
B = {rs, sr}

\end{document}